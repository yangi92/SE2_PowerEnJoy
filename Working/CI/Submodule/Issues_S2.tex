% Issues list for simone
\subsubsection{Naming conventions}
\begin{itemize}
\item All class names, interface names, method names, class variables, method
variables, and constants used have meaningful names and do
what the name suggests. (nqdbl, nqint, curpcms, curpcm che cazzo vor dì?) \\

\item The only one-character variable (Exception e) is a temporary
“throwaway” variable.\\

\item Class names are nouns, in mixed case, with the first letter of each word
in capitalized.\\

\item There are no interfaces.\\

\item Method names are verbs, with the first letter of each addition
word capitalized.\\

\item All attributes, are mixed case, no one begin
with an underscore. All the
remaining words in the variable name have their first letter capitalized, except one (cartiter, which should have been cartIter).
 \\
 
 \item There are no constant.\\
\end{itemize}

\subsubsection{Indention}
\begin{itemize}
\item Four spaces are used for indentation and it is done consistently. \\
\item No tabs are used to indent.\\
\end{itemize}


\subsubsection{Braces}
\begin{itemize}
\item In all the class is used the “Kernighan and
Ritchie” style (first brace is on the same line of the instruction that
opens the new block).\\

\item All while, do-while, try-catch, and for statements that have
only one statement to execute are surrounded by curly braces. There are 10 if statements which contain onlu one statement and are written in one line without braces.\\

\end{itemize}

\subsubsection{File Organisation}
\begin{itemize}
\item Either blank lines or comments are used to separate sections.\\
\item Line length exceeds 80 characters many times. 10 times it exceeds also 120 characters One of them it's over 200.\\ 
\end{itemize}

\subsubsection{Wrapping Lines}
\begin{itemize}
\item Line break occurs only after a comma or an operator. \\
\item ???????????? Higher-level breaks are used ????????????????????? \\
\item A new statement is aligned with the beginning of the expression at the
same level as the previous line.\\
\end{itemize}


\subsubsection{Comments}
\begin{itemize}
\item Comments are used to adequately explain what the class, interface,
methods, and blocks of code are doing. \\
\item ??????????????????????????????????????????????? Commented out code contains a reason for being commented out and
a date it can be removed from the source file if determined it is no
longer needed.????????????????????????????????????????????????
\end{itemize}





