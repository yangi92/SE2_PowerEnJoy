% Issues list for simone
\subsubsection{Naming conventions}
\begin{itemize}

\item Line 268: nqdbl has no meaningful name
\item Line 208: nqint has no meaningful name
\item Line 84: curpcms has no meaningful name
\item Line 86: curpcm has no meaningful name

\item Line 137: the only one-character variable (Exception e) is a temporary
“throwaway” variable.\\

\item All class names are nouns, in mixed case, with the first letter of each word
in capitalized.\\

\item There are no interfaces.\\

\item All method names are verbs, with the first letter of each addition
word capitalized.\\

\item All attributes, are mixed case, no one begin
with an underscore. All the
remaining words in the variable name have their first letter capitalized, except one (cartiter, which should have been cartIter).
 \\
 
 \item There are no constants.\\
\end{itemize}

\subsubsection{Indention}
\begin{itemize}
\item Four spaces are used for indentation and it is done consistently. \\
\item No tabs are used to indent.\\
\end{itemize}


\subsubsection{Braces}
\begin{itemize}
\item In all the class is used the “Kernighan and
Ritchie” style (first brace is on the same line of the instruction that
opens the new block).\\

\item All while, do-while, try-catch, and for statements that have
only one statement to execute are surrounded by curly braces. There are 10 if statements which contain onlu one statement and are written in one line without braces.\\

\end{itemize}

\subsubsection{File Organisation}
\begin{itemize}
\item Either blank lines or comments are used to separate sections.\\
\item Line length exceeds 80 characters many times:
\begin{itemize}
\item Line 61 
\item Line 64
\item Line 70
\item Line 76
\item Line 79
\item Line 81
\item Line 82
\item Line 83
\item Line 84
\item Line 87
\item Line 88
\item Line 89
\item Line 99
\item Line 100
\item Line 101
\item Line 116
\item Line 117
\item Line 121
\item Line 148
\item Line 151
\item Line 154
\item Line 158
\item Line 159
\item Line 160
\item Line 162
\item Line 165
\item Line 168
\item Line 169
\item Line 170
\item Line 175
\item Line 176
\item Line 178
\item Line 183
\item Line 185
\item Line 187
\item Line 190
\item Line 192
\item Line 193
\item Line 194
\item Line 198
\item Line 203
\item Line 204
\item Line 208  
\item Line 210
\item Line 214
\item Line 215
\item Line 216
\item Line 220
\item Line 221
\item Line 225
\item Line 237
\item Line 239
\item Line 241
\item Line 256
\item Line 260
\item Line 261
\item Line 262
\item Line 265
\item Line 268
\item Line 288
\item Line 295
\item Line 303
\item Line 332 
\end{itemize}

\item 18 times it exceeds also 120 characters:
\begin{itemize}
\item Line 76
\item Line 77
\item Line 79
\item Line 84
\item Line 117
\item Line 121
\item Line 158
\item Line 159
\item Line 160
\item Line 169
\item Line 198
\item Line 214
\item Line 215
\item Line 216
\item Line 237
\item Line 260
\item Line 268
\item Line 288
\end{itemize}
\end{itemize}

\subsubsection{Wrapping Lines}
\begin{itemize}
\item Line break occurs only after a comma or an operator. \\
\item Higher-level breaks are not used. \\
\item A new statement is aligned with the beginning of the expression at the
same level as the previous line.\\
\end{itemize}


\subsubsection{Comments}
\begin{itemize}
\item Comments are used to adequately explain what the class, interface,
methods, and blocks of code are doing. \\
\item No issues found about commented out code. \\
\end{itemize}
