% Issues list for simone

\subsubsection{Naming conventions}
\begin{itemize}

\item Line 99:  The method setCurrencyUomIfNone doesn't have a meaningful name. 
 
\item Line 50: The only one-character variable (Exception e) is a temporary
“throwaway” variable.

\item All class names are nouns, in mixed case, with the first letter of each word
in capitalized.

\item There are no interfaces.

\item All method names are verbs, with the first letter of each addition
word capitalized.

\item All attributes, are mixed case, no one begin
with an underscore. All the
remaining words in the variable name have their first letter capitalized.
 
 
 \item There are no constants.
\end{itemize}

\subsubsection{Indention}
\begin{itemize}
\item Four spaces are used for indentation and it is done consistently. 
\item No tabs are used to indent.
\end{itemize}

\subsubsection{Braces}
\begin{itemize}
\item In all the class is used the “Kernighan and
Ritchie” style (first brace is on the same line of the instruction that
opens the new block).

\item All if, while, do-while, try-catch, and for statements that have
only one statement to execute are surrounded by curly braces. 

\end{itemize}

\subsubsection{File Organisation}
\begin{itemize}
\item Either blank lines or comments are used to separate sections.
\item Line length exceeds 80 characters many times:
\begin{itemize}
\item Line 42 
\item Line 44 
\item Line 49 
\item Line 60 
\item Line 66 
\item Line 76 
\item Line 78 
\item Line 84 
\item Line 86 
\item Line 88 
\item Line 91 
\item Line 94 
\item Line 97 
\item Line 98 
\item Line 99 
\item Line 100
\item Line 101
\item Line 103
\item Line 104
\item Line 105
\item Line 108
\item Line 110
\end{itemize} 
\item Seven times it exceeds also 120 characters: 
\begin{itemize}
\item Line 78
\item Line 84
\item Line 88
\item Line 94
\item Line 97
\item Line 98
\item Line 110
\end{itemize}
\end{itemize}

\subsubsection{Wrapping Lines}
\begin{itemize}
\item Line break occurs only after a comma or an operator. 
\item Higher-level breaks are not used. 
\item A new statement is aligned with the beginning of the expression at the
same level as the previous line.
\end{itemize}


\subsubsection{Comments}
\begin{itemize}
\item Comments are used to adequately explain what the class, interface,
methods, and blocks of code are doing. 
\item No issues found about commented out code. 
\end{itemize}





