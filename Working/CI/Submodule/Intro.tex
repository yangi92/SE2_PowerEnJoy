The code inspection is a systematic examination of computer source. It is intended to find mistakes overlooked during the initial development phase, with the aim of developing the overall quality of the software.\\
This document indexes all coding mistakes found following a standard code inspection check-list for the requested classes listed in the underling section \ref{sub:Elenco}\\
\subsection{Class overview}
\label{sub:Elenco}
The Apache OFBiz Project is an open source product for the automation of enterprise processes that includes framework components and business applications for ERP,CRM and other business-oriented functionalities.\footnote{Full source code: \url{http://mirror.nohup.it/apache/ofbiz/apache-ofbiz-16.11.01.zip}}\\
Among the many, the following classes were assigned for code inspection: 
\begin{itemize}
\item \textbf{ProductStoreCartAwareEvents.java}
\begin{itemize}
\item \textit{Location}: apache-ofbiz-16.11.01/applications/order/src/main\\/java/org/apache/ofbiz/order/shoppingcart/product\\/ProductStoreCartAwareEvents.java
\item \textit{Class role}: see section \ref{sub:PSCAE}
\item \textit{Total number of issues found}: TBD
\end{itemize}
\item \textbf{ProductDisplayWorker.java}
\begin{itemize}
\item \textit{Location}: apache-ofbiz-16.11.01/applications/order/src/main\\/java/org/apache/ofbiz/order/shoppingcart/product\\/ProductDisplayWorker.java
\item \textit{Class role}: see section \ref{sub:PDW}
\item \textit{Total number of issues found}: TBD
\end{itemize}
\end{itemize}