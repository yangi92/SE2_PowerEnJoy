\subsection{Size estimation}
The final size of the project will be estimated in the  LOC measure throught the use of Function Points.
The following table show the weight assigned to each function points:\\

\import{Tables/}{WeightFunctionPoints.tex}

\subsubsection{Inputs}

\import{Tables/}{Input.tex}

\subsubsection{Output}
\import{Tables/}{Output.tex}

\subsubsection{Inquiries}
\import{Tables/}{Inquiries.tex}

\subsubsection{Internal Logic Files}
\import{Tables/}{ILF.tex}

\subsubsection{External Logic Files}
\import{Tables/}{ELF.tex}

\subsubsection{Total number of UFPs}
\import{Tables/}{TotalFPs.tex}

\subsection{Effort and cost estimantion}
\subsubsection{COCOMO II}
The \emph{COCOMO II} model is an evolution from the \emph{COCOMO 81} and is used to express \textbf{effort} as \emph{PERSON-MONTHS}.\\
In particular it uses the following formula to estimate the total \textbf{effort} :
\begin{equation}
    \textrm{Effort} = A \times SIZE^E \times \prod_i EM_i
    \label{eq:effort}
\end{equation}
where :
\begin{itemize}
	\item $A$ is given statistically and is equal to $2.94$
	\item $SIZE$ is the size of the software expressed in KLOC
	\item $E$ is an aggregation of five scale factors (SF) 
	\item $EM$ are the \emph{effort multipliers} of the \emph{cost drivers} 
\end{itemize}
In the follwing sections we will calculate all the parameters mentioned above, in order to generate the final result of the formula.

\subsubsection{Scale factors estimation}
\label{sub:scale_factors}
This section provides the estimation for the scale factors.

\import{Tables/}{ScaleDriversExt.tex}



\subsubsection{Cost drivers effort multipliers estimation}
\label{sub:cost_drivers}
This section provides the estimation for the effort multipliers of the cost drivers.

\import{Tables/}{EMext.tex}

\subsubsection{Final effort estimation}
\label{sub:effort_estimation}
Now that we have all the parameters of the formula, we can calculate the final result:
\begin{itemize}
	\item $A = 2.94$
	\item $SIZE = 81UFP  \times  53  = 4.293  KLOC$ ($53$ is the JAVA multiplier)
	\item $\prod_i EM_i = 0.624$
	\item $E = 1.038$
\end{itemize}

\begin{equation}
    \textrm{Effort} = A \times SIZE^E \times \prod_i EM_i = 8.32 PM
    \label{eq:effort}
\end{equation}
So the effort to develop the project is 8.32 person-months.\\
We are 3 people, so this means less than 3 months development. 
