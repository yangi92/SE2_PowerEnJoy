\subsection{Entry Criteria}
The \textbf{Integration tests} are meant to be developed and conducted only after \textbf{single units} have been successfully and thoroughly tested ,with particular regard towards those parts involving intermodule communication.\\


\subsection{Elements to be Integrated}
From what we can infer from the previous documents , the to-be tested platform
used the \textbf{client-server paradigm} as its main architecture with the addition of intra module communication , especially in the \textbf{back-end system} where the \textit{business logic} lies, and direct communication happening back and forth on separate channels between the \textbf{back-end} and the \textbf{client-side applications.} \\
The following components described in \emph{section 2} of the \emph{Design Document} need to be tested:
\begin{itemize}
\item \textbf{Server Components}:
\begin{itemize}
\item Ride Manager
\item Notification Manager
\item User Manager
\item Search Manager
\item Position Manager
\item Database Interface
\end{itemize}
\item \textbf{Mobile \& Web Application}
\begin{itemize}
\item SignIn Action
\item SignUp Action
\item Main Action
\end{itemize}
\item \textbf{On-Board Application}
\begin{itemize}
\item SignIn Action
\item Navigation Action
\item EndRide Action
\end{itemize}
\end{itemize}

The client-side applications communicate with the \emph{back-end system} through the \textbf{RESTful API} and the \textbf{Notification Manager}. To simply test planning, from now on \emph{Mobile \& Web components} and \emph{On-Board components} are grouped together in a single entity called \textbf{client-side components}.With this considerations in mind , integration test need to be performed on the following pairs:
\begin{itemize}
\item Client-side components $\rightarrow$ RESTful API
\item RESTful APi $\rightarrow$ User Manager
\item RESTful API $\rightarrow$ Search Manager
\item RESTful API $\rightarrow$ Position Manager
\item RESTful API $\rightarrow$ Ride Manager
\item Position Manager $\rightarrow$ Vehicle Manager
\item User Manager $\rightarrow$ DBMS 
\item Ride Manager $\rightarrow$ DBMS
\item Vehicle Manager $\rightarrow$ DBMS
\item Search Manager $\rightarrow$ Notification Manager
\item Ride Manager $\rightarrow$ Notification Manager
\item Notification Manager $\rightarrow$ Client-side components
\item \textbf{STATION MANAGER!!!!!!!!!!!!!!!!!!!!!!!!!!!!!!!!!!!!!!!!!!!!!!}
\end{itemize}

\subsection{Integration Testing Strategy}
The \emph{PowerEnjoy System} is composed of many components which are subject to a lot of interactions : the system , as already shown in the Design Document, is thus quiet \textbf{complex}. Structural testing strategies , such as top-down or bottom-up, are \emph{simpler} whereas more complex strategies provide better process visibility in cases like ours.\\
The strategy we adopted is the \textbf{functional grouping strategy} , a highly \emph{modular} strategy allows the \emph{separate} development of the various parts of the system.\\
Moreover the integration test should be performed mostly on \emph{actual code} in order to reduce the number of \emph{stubs and mocks} and reduce the use of \emph{dummy code}.

\subsection{Integration Sequence}
The \textbf{Integration Test} is meant to be performed in the following order in compliance to the aforementioned  strategy:
\begin{enumerate}
\item Server application components
\item Client applications components
\item Client-Server
\end{enumerate}






