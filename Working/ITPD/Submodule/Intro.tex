\subsection{Purpose and Scope}
The purpose of the \textbf{Integration Test Plan} is to describe the necessary tests to verify that all of the components of the \emph{PowerEnjoy} platform are properly assembled. Integration testing ensures that the unit-tested modules interact correctly.\\
The description of the testing process includes:
\begin{itemize}
\item A high level specific of the tests
\item A testing strategy
\item An overview of the testing tools
\end{itemize}
The document is aimed at stakeholders ,developers in charge of the testing implementation and engineers.\\
It is important to notice that the focus of the document lies essentially towards \textbf{integration} whereas \textbf{unit-tests} are ignored and considered as already conducted. 
\subsection{Definition and Abbreviations}
Throughout the document the following \textit{abbreviations} are used and not further explained:
\begin{itemize}
\item \textbf{RASD}: Requirements And Specifications Document
\item \textbf{DD}: Design Document
\item \textbf{ITPD}: Integration Test Plan Document
\item \textbf{API}: Application Programming Interface
\item \textbf{RESTful}:REpresentational State Transfer
\end{itemize}
Each \textbf{integration test} has a unique identifier that follows the syntax: $$ I[0-9]^+$$
Each \textbf{test case} has a unique identifier that follows the syntax: $$I[0-9]^+T[0-9]^+$$
 
\subsection{Reference Documents}
For a full understanding of the content of the ITPD ,it is strongly advised to read the \textbf{RASD} and especially the \textbf{DD} as they contain more in-depth explanations for the majority of the subjects.\\
A complete overview about documents and the general system description can be found int the \textbf{Assignments AA 2016-2017.pdf} file.



