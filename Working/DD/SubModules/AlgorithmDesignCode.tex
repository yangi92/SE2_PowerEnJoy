\newcommand\tab[1][1cm]{\hspace*{#1}}

\subsection{Car Search}
The Search and Position Manager provide all the methods to handle incoming search queries. The incoming HTTPS request is dispatched by 
the request manager which has to decide the manager it needs to forward the request.\\
\begin{itemize}
\item \textbf{carSearch(SearchData)}: the user is asked to input a desired location and radius within which the search takes place.The resulting data will be collected inside a \emph{Search Data object} which will be forwarded inside a \emph{SearchAction} to the \emph{Back-End} through the \emph{carSearch()} method.

\item \textbf{getAvailableVehicles()}: after the \emph{back-end} identifies correctly the request through the \emph{request manager}, the \emph{Search Manager}
extracts the data inside the request and queries the \emph{Position Manager} for the available cars.\\
Through the \emph{Notification Manager} the available car positions will be send back to the \emph{client-side} who will then take car to display the results.
\end{itemize}

\subsection{Car Reservation}
After a \emph{Search} has been made the \emph{User} selects the \emph{vehicle} he desires to reserve. The \emph{selected car} will be encapsulated inside a \emph{Ride Action object} which will be send to the \emph{back-end}.Once the \emph{Ride request} has been dispatched correctly  the \emph{Ride Manager} creates a \emph{Ride Object} and starts the \emph{Timer}.\\The \emph{Ride Manager} listens in the background to check to \emph{user position}. Once the user is near the desired car , a \emph{unlock-notification} will be send to the \emph{user-client}.

\begin{itemize}
\item \textbf{reserveCar(Car car)}: encapsulates the relevant \emph{car data} inside a \emph{RideAction} and forwards it to the \emph{back-end}.

\item \textbf{unlockCar()}: once a notification has been received \emph{client-side} , this method trigged through an UI Component (like a button) unlocks the car and stops the timer.
\end{itemize}

\subsection{Ride \& Park}
The ride and park action are handled mainly by the \emph{Ride Manager}.The following actions can be performed only after a successful \emph{unlock-action}\\
\begin{itemize}
\item \textbf{authenticate(User user, int pinCode)} : the \emph{On-Board application}requires to input the user code to begin the ride.If the \emph{PinCode} corresponds to the one assigned to the User,the authentication process is completed and the user can start to use the \emph{on-board application}.

\item \textbf{startRide(NavigationData,Route)}:once the user has selected one of the \emph{routes} suggested by the system through \emph{Google Maps API} queries, a
\emph{RideData Object}is created to keep track of time, distance, battery level and bonuses. During the route selection process the \emph{Money Saving Option} can be enable through the toggle of an UI Component Button.Car sensors check for passengers in car which will be added as \emph{statistic} and later considered during bonus assignment.

\item \textbf{parkCar()}: if the car is \emph{standing still} and in \emph{parking modus} , the \emph{on-board display} allows to \emph{end the ride} via an UI Component (like a button).\\
Pressing the button triggers the \emph{parkCar()}method which will send a request to the \emph{back-end application} to check if the car is parked in a \emph{Safe Area}.

\item \textbf{lockCar(RideData rideData)}:once the \emph{parkCar()} method has successfully 
send a positive response to the \emph{on-board application} the system waits for the user to leave the car and close the doors.\\
If the car is empty and the doors are closed a signal-message is send to the \emph{back-end application} to \emph{lock the doors} and \emph{update the car status}.\\Subsequently the \emph{RideData} will be forwarded to the \emph{back-end application} that updates the \emph{Ride object} with the final data and calculates the bonuses and final fee.
\end{itemize}

\subsection{Low Level Algorithm Description}
\subsubsection{Search}

\begin{lstlisting}

/* performed inside the Search Manager */
getAvailableCars(SearchData data){
	List<VehicleManager> nearbyCars = new ArrayList<VehicleManager>;
	Position p = data.getPosition();
	Radius r= data.getRadius();
	nearbyCars=queryPosManger(p,r);
	notificationMangager.send(data.getUserDevice);

}

/* performed by the Position Manager */
getAvailableCarList(Position p,Radius r){
	List<VehicleManager> nearbyCars= new ArrayList<VehicleManager>;	
	for(VehicleManager v: Vehicles){
		if (distance(v.position,p)<r)
		nerabyCars.add(v);
	}
}

/* performed client-side after receiving the notification */
...
navigationAction.displayMap(cars)
...

displayMap(List<VehicleManager> cars){
	Map map = GoogleMaps.getMap();
	map.render();
	for(VehicleManager c: cars){
		Position p=c.getPosition();
		mapIndicator = new MapIndicator(p);
		mapIndicator.render();
	}
}


 

\end{lstlisting}

\subsubsection{Reserve}
\begin{lstlisting}

/* performed client-side */
reserveCar(Car car){
 	action=new RideAction(encapusulateCarData(car));
 	sendRequest(action);
}


/* performed server-side inside the Ride Manager*/

createRide(RideData data){
	if(data.isConsistent()){
		ride= new Ride();
		/* all ride data added battery status,position,car...*/
		posManager.updateStatus(data.getCar(),newBusyCarStatus));
		ride.startTimer();
		notificationManager.send(Message,data.getUserDevice());	
	}
	else notificationManager.send(Message,data.getUserDevice());
}



\end{lstlisting}
\subsubsection{Unlock}
\begin{lstlisting}

/* Server-Side inside Ride Manager*/
checkIfReadyToUnlock(Ride r, User u){
	while(!(isTimeUp())){
	  if(distance(ride.getCar().getPosition(),u.currentPosition())<0.002){
			notificationManager.send(new UnlockAction(...),u.getDevice);
		}
	}
}
\end{lstlisting}


\subsubsection{Authenticate \& StarRide}
\begin{lstlisting}
/* On-board authentication request */

authenticate(User u,int pinCode){
	request =new AuthenticationRequest(pinCode,u);
	request.encode();
	sendRequest(u.getDevice, request);
	reponse= getInputStream();
	if(response.valid()){
	 notificationManager(display(response.message));	
	 updateCarStatus(new DrivingCarStatus);
	}
	else{
	   notificationManager(display(response.message));
	}
	
}

startRide(Position destination, Preferences pref){
	List<Route> routes = new ArrayList<>();
	if(pref.getMSOFlag()){
		destination = getNewMSOPosition(destination));	
	}	
	routes.add(GoogleMaps.getPathes(destination,carPosition));
	displayMap();
	displayRoutes(routes);
}

/* The getMSOPostion method is inside navigationAction class and forwards data to the back-end.
 */
 
 getMSOPosition(Position destination){
	sendMSORequest(destination);
	/* the back-end checks within the database for the closest charging station and return the position to the on-board application */
	reponse=getInputStream();
	return response.getNewDestination(); 
 }

\end{lstlisting}

\subsubsection{Park \& Lock}
\begin{lstlisting}

/* on-board side*/
parkCar(){
	sendParkingRequest(rideData);
	response=getInputStream();
	if(response.isPositive());
		{
		  notificationManager.display(Message);
		  	 updateCarStatus(new ParkedCarStatus);
		}
	else{
		notificationManager.display(Message);
	}
}

/*Server-side inside the Ride Manager class*/

checkSafeArea(RideData r){
	 position=r.getCurrentPostion();
	 if(isSafAreaPostion(position);
	 /* create response object */
	 notificationManager.send(response,r.getCarDevice());
}

/* Car locking is done server-side.As soon as the sensors detect an empty car in parked status ,the lockCar mehtod gets triggered server-side. */

lockCar(RideData r){
	ride.status = end;
	ride.timer.stop();
	car.lock();
	float fee = ride.calculateFee();
	if (ride.numberPassenger >=2)
		totalFee = ride.fee - 0,1*ride.fee;
	if (car.batteryLevel> 0,5)
		totalFee = ride.fee - 0,2*ride.fee;
	if (car.isInCharge == true)
		totalFee = ride.fee - 0,3*ride.fee;
	if (car.position is not in powerGrid or car.batteryLevel<0,2)
		totalFee = ride.fee + 0,3*ride.fee;
	car.status = available;
	ride.fee = totalFee;
	return true;
}

\end{lstlisting}





